\documentclass{article}
\usepackage[utf8]{inputenc}
\usepackage[T1]{fontenc}
\usepackage{amsmath, amsthm}
\usepackage{amsfonts}
\usepackage{algorithm}
\usepackage{algorithmic}
\usepackage{float} % 为 algorithm 的 [H] 选项提供支持
\usepackage[utf8]{inputenc}
\usepackage[T1]{fontenc}
\usepackage{xcolor}
\usepackage[colorlinks=true, urlcolor=red]{hyperref}
\usepackage{enumitem}
\usepackage{tikz}
\usetikzlibrary{positioning}

\newtheorem{assumption}{Assumption}
\newtheorem{lemma}{Lemma}
% 定义 theorem 环境(按章节编号)
\newtheorem{theorem}{Theorem}[section]
% 以下环境共用 theorem 的编号
\theoremstyle{definition}
\newtheorem{definition}[theorem]{Definition}
\newtheorem{example}[theorem]{Example}
\newtheorem{remark}[theorem]{Remark}
\newtheorem{corollary}[theorem]{Corollary}
\newcommand{\Ito}{It\^o}
\newcommand{\sgn}{\operatorname{sgn}}
% 如果希望 Proposition 与 theorem 共享编号,则使用下面的语句
% \newtheorem{proposition}[theorem]{Proposition}
% 如果希望 Proposition 独立编号,则使用下面的语句
\newtheorem{proposition}{Proposition}
\title{Final Project Report: Building GPT-2}
\date{\today}

\begin{document}
    \maketitle
    \tableofcontents
  \section{Basic information}


  The proposal should have the following information at the top:
  \begin{itemize}
    \item \textbf{Title:} A Framework for Building and Fine-Tuning GPT-2 Models
    \item \textbf{Team members:} 
    \begin{itemize}
      \item Songlin Songlin (\texttt{zhousl24@tsinghua.edu.cn})
      
      \textbf{Contribution}: Doing theoratical research and writing project proposal, 
implementing the structure of GPT-2 within the process of
\href{https://github.com/AGIXLab/NLU-Course-Project-GPT-and-Downstream-Tasks/tree/main}{Stanford NLU Final Project Guidence},
which is the first part of our project.


      Zhongwei Sun(\texttt{sun-zw23@mails.tsinghua.edu.cn})

      Shibo Dai(\texttt{dsb24@mails.tsinghua.edu.cn})
    \end{itemize}
    \item \textbf{Custom or Default Project:} 
      Default Project---Building GPT-2
      % 若使用默认模板则填 “Default Project”
  \end{itemize}
  





  



  \section{Project description}

  \begin{itemize}
    \item \textbf{Main Goals:} 
      \begin{itemize}
        \item Use the codebase that is adapted from the final project of Stanford CS224n with pre-trained of GPT-2
        and implement some important components of the GPT-2 model to better understand its architecture.
        See \href{https://github.com/AGIXLab/NLU-Course-Project-GPT-and-Downstream-Tasks/tree/main}{this link}.
        \item Fine-tune the pretrained model on multiple downstream NLP tasks: 1)
        Sentiment Analysis, 2) Paraphrase Identification, and 3) Sonnet Generation.
        \item Train a mini-GPT model from scratch without loading any pretrained transformer weights (word embeddings are allowed).
      \end{itemize}
  
    \item \textbf{Downstream NLP Tasks:}
      \begin{itemize}
        \item Sentiment Analysis on IMDB reviews (binary classification).
        \item Paraphrase Detection on MRPC (sentence-pair classification).
        \item Sonnet Generation given a short Shakespearean prompt (autoregressive generation).
      \end{itemize}
  
    \item \textbf{Data Usage:}
      \begin{itemize}
        \item \emph{Pre-training:} None. The pre-trained data is provided within \emph{
          NLU-Course-Project-GPT-and-Downstream-Tasks}


        \item All we need to do is complete the implements of the missing block of algorithm such as Multi-headed Attention,
        Adam Optimization and position-wise Feed-Frward Network.
        \item \emph{Fine-tuning and Prediction for Sentiment Analysis:}
          \begin{itemize}
            \item Stanford Sentiment Treebank (SST) (8,544 train / 1,101 dev / 1,101 test).
            \item CFIMDB dataset (1,701 train / 245 dev / 488 test)
          \end{itemize}
        \item \emph{Fine-tuning and Prediction for Paraphrase Detection:}
          \begin{itemize}
            \item Quora dataset (141,506 train  / 20,215 dev / 40,431 test)
          \end{itemize}
        \item Test sets and methods are also provided in the e final project of
        Stanford CS224n.
      \end{itemize}
  
    \item \textbf{Methods:}
      \begin{itemize}
        \item Use the GPT-2 structure provided within \emph{
          NLU-Course-Project-GPT-and-Downstream-Tasks}: GPT-2 small (12 layers, 12 heads, 768-dim).
        \item Training with Adam, linear warmup (10 \% steps) + linear decay.
        \item Fine-tuning: add classification heads for IMDB/MRPC; use beam search for sonnet generation.
      \end{itemize}
  
    \item \textbf{Baselines:}
      \begin{itemize}
        \item For SA:
          \begin{itemize}
            \item Last Linear Layer for SST: Dev Accuracy: 0.462
            \item Full Model for SST: Dev Accuracy: 0.513
            \item Last Linear Layer for CFIMDB: Dev Accuracy: 0.861
            \item Full Model for CFIMDB: Dev Accuracy: 0.976
          \end{itemize}
        \item For PD:
        \begin{itemize}
          \item \textbf{IMDB Sentiment Analysis:}
              \item BERT‐base fine‐tuned (Devlin et al., 2019): 94.5\% accuracy.
              \item GPT-2 zero‐shot (Radford et al., 2019): 70.3\% accuracy.

            \item \textbf{MRPC Paraphrase Detection:}
              \item BERT‐base fine‐tuned (Devlin et al., 2019): 88.9\% accuracy, F1 = 89.2\%.
              \item GPT-2 zero‐shot (Radford et al., 2019): 65.1\% F1.
        \end{itemize}
        \item For Sonnet: Published BERT-base fine-tuned results (GLUE leaderboard).
      \end{itemize}
  
    \item \textbf{Evaluation Metrics:}
      \begin{itemize}
        \item See Baselines for SA.
        \item \emph{Accuracy} and \emph{F1} for IMDB and MRPC.
        \item \emph{CHRF score} for sonnet generation.
      \end{itemize}
  \end{itemize}

  \section{Songlin Zhou: Implementing GPT-2}
  \subsection{Folder and File Roles}

  \begin{description}
    \item[\texttt{\_\_pycache\_\_}]%
          A local snapshot of the \emph{official} HuggingFace
          GPT{\tiny -}2 weights; it is loaded on demand by
          \texttt{base\_gpt.py} to avoid an on-line download.
  
    \item[\texttt{Downstream-tasks/}]%
          Example fine–tuning code for text classification, summarisation,
          \emph{etc.}\;—\;not required in my assignment.
  
    \item[\texttt{models/}]  \hfill (central folder)
            \begin{description}
              \item[\texttt{\_\_pycache\_\_}]%
                     Compiled bytecode and cached tensors; reuse of the
                     pretrained weights.
  
              \item[\texttt{base\_gpt.py}]%
                     Generic “\emph{foundation}’’ class
                     (\texttt{GPTPreTrainedModel})\,: stores the
                     \texttt{config}, performs weight initialisation
                     and keeps track of global \texttt{dtype}. Every
                     concrete GPT variant inherits from this class.
  
              \item[\texttt{gpt2.py}]%
                     Instantiates the Transformer stack defined in
                     \texttt{base\_gpt.py}, wiring the attention blocks,
                     feed–forward layers, and the final LM head.
            \end{description}

    \item[\texttt{modules/}]
    

            \begin{description}
              \item[\texttt{attention.py}]%
                     Full implementation of \emph{Causal Self-Attention}
                     (multi–head, padding and causal masking, output
                     projection).
    
              \item[\texttt{gpt2\_layer.py}]%
                     Combines \texttt{attention.py} with
                     \emph{Pre-LayerNorm} and the feed-forward MLP to
                     form one Transformer block
                     (\texttt{Dropout $\rightarrow$ Dense $\rightarrow$
                     Residual} per layer).
          \end{description}
  
    \item[\texttt{test/}]%
          Minimal unit tests for GPT-2 model forward-pass correctness and for the
          \texttt{Adam} optimiser.
          \begin{verbatim}
            $ cd my-gpt2-project
            $ python3 test/optimizer_test.py
            Your GPT2 implementation is correct!
            $ python3 test/sanity_check.py
            Optimizer test passed!
            \end{verbatim}
            

          The following steps demonstrate the verification process for the GPT-2 implementation and optimizer:
  
    \item[\texttt{config.py}]%
          Default hyper-parameters (hidden size, \#layers, dropout
          probability, learning rate, \dots).
  
    \item[\texttt{optimizer.py}]%
          A clean implementation of the \texttt{Adam} algorithm.
    \item[\texttt{README.md}]%
          Step-by-step running instructions.
  \end{description}

  \begin{center}
    \fbox{\parbox{0.9\textwidth}{
        \centering
        \textbf{Code Repository} \\
        \url{https://github.com/Pinarinith/my-gpt2-project}
    }}
    \end{center}

  \subsection{Attention Module (\texttt{attention.py})}

\begin{description}
  \item[\texttt{def \_\_init\_\_(self, config):}] 
    Initializes the three linear projection layers for \textbf{query}, \textbf{key} and \textbf{value}, each of shape $[H]\!\to\![H]$, where $H=\text{hidden\_size}$.  Imports all relevant hyperparameters from \texttt{config} (e.g.\ \texttt{num\_attention\_heads}, \texttt{hidden\_size}, \texttt{attention\_dropout}).  Defines a \texttt{Dropout} layer that is applied to the \emph{normalized} attention scores following the original Transformer paper.  Although somewhat unusual, we empirically observe that this yields better generalisation.

  \item[\texttt{def transformer(self, x, linear\_layer):}]
    Given an input feature tensor 
    \[
      x\in\mathbb{R}^{B\times T\times H},
    \] 
    where $B$ is batch size, $T$ is sequence length and $H$ is hidden dimension, applies a single linear projection (\texttt{linear\_layer}) to obtain 
    \(\displaystyle \mathrm{proj}\in\mathbb{R}^{B\times T\times H}\), then reshapes into multiple heads:
    \[
      \text{proj}
        =\texttt{rearrange}(\text{proj},\,`\;b\;t\;(h\;d)\to b\;t\;h\;d',\;h=\text{num\_attention\_heads}),
    \]

    which yields a tensor of shape $[B,T,h,d]$ and enables parallel computation over $h$ attention heads of size $d=H/h$ for speed and efficiency.

  \item[\texttt{def attention(self, key, query, value, attention\_mask):}]
    Computes masked, scaled dot–product self–attention.  Given
    \[
      Q,K\in\mathbb{R}^{B\times h\times T\times d},\quad
      V\in\mathbb{R}^{B\times h\times T\times d},
    \]
    it first forms the unnormalized scores
    \[
      S = \frac{Q\,K^{\!\top}}{\sqrt{d_k}}\quad\Bigl[S\in\mathbb{R}^{B\times h\times T\times T}\Bigr],
    \]
    then applies the mask:
    \[
      S \leftarrow S + \text{attention\_mask},
    \]
    and finally normalises with a dropout to keep robustness:
    \[
      A = \mathrm{softmax}(S,\;\mathrm{dim}=-1),\quad
      A = \mathrm{Dropout}(A).
    \]
    The output context is
    \[
      \text{context} = A\,V\in\mathbb{R}^{B\times h\times T\times d},
    \]
    which is reshaped back to $[B,T,H]$ before returning.

    \item[\texttt{def forward(self, hidden\_states, attention\_mask):}]
    In the feed‐forward neural network we apply masking 
    of future information to prevent the model 
    from being influenced by future tokens during training, 
    which would degrade prediction performance. 
    Each \texttt{forward} call outputs an attention value 
    that becomes a subcomponent in \texttt{gpt2\_layer.py}.
    
\end{description}

\subsection{Layer Module (\texttt{gpt2\_layer.py})}

\begin{description}
  \item[\texttt{def \_\_init\_\_(self, config):}] 
    Initializes all sub–modules for one Transformer block:  
    \begin{itemize}
      \item Multi–head self–attention \(\bigl(\texttt{CausalSelfAttention}\bigr)\) parameters  
      \item Two linear layers for the position‐wise feed–forward network  
      \item Two \texttt{LayerNorm} instances for Pre–LN  
      \item Dropout probabilities from \texttt{config}  
    \end{itemize}

  \item[\texttt{def add(self, residual, sublayer\_out, dense\_layer, dropout):}]  
    \begin{itemize}
      \item Projects \texttt{sublayer\_out} back to the hidden dimension via \texttt{dense\_layer}.  
      \item Applies \texttt{dropout} for regularisation.  
      \item Adds the original \texttt{residual} tensor (residual connection).  
    \end{itemize}

  \item[\texttt{def forward(self, hidden\_states, attention\_mask):}]  
    Implements one full block in the Pre–LayerNorm style:
    \begin{enumerate}
      \item \textbf{Multi–Head Self–Attention}
        \begin{enumerate}
          \item[(1‐a)] \emph{Pre‐normalize}: apply \texttt{LayerNorm} to \texttt{hidden\_states}.
          \item[(1‐b)] \emph{Attention}: compute self–attention with \texttt{attention\_mask}.
          \item[(1‐c)] \emph{Dropout + Residual}: use \texttt{add(...)} to project, drop out, and add back the input.
        \end{enumerate}
      \item \textbf{Position‐wise Feed–Forward}
        \begin{enumerate}
          \item[(2‐a)] \emph{Pre‐normalize}: apply \texttt{LayerNorm} to the attention output.
          \item[(2‐b)] \emph{MLP}: apply \texttt{Linear} → \texttt{GELU} → \texttt{Linear}.
          \item[(2‐c)] \emph{Dropout + Residual}: use \texttt{add(...)} again to project, drop out, and add back.
        \end{enumerate}
    \end{enumerate}
\end{description}

  
\subsection{GPT2 Module Summary (\texttt{gpt2.py})}

\begin{description}

  \item[\texttt{def \_\_init\_\_(self, config):}]  
    Initializes embeddings, Transformer blocks, layer‐norm and weights.

  \item[\texttt{def embed(self, input\_ids):}]  
    Token + position lookup + dropout.

  \item[\texttt{def encode(self, hidden, mask):}]  
    Applies each layer with the extended attention mask.

  \item[\texttt{def forward(self, input\_ids, attention\_mask):}]  
    Embed → encode → final norm → select last token.
\begin{verbatim}
x    = self.embed(input_ids)
x    = self.encode(x, attention_mask)
x    = self.ln_f(x)
last = x[torch.arange(B), attention_mask.sum(1)-1]
return {'last_hidden_state': x, 'last_token': last}
\end{verbatim}

  \item[\texttt{def hidden\_state\_to\_token(self, h):}]  
    Weight–tie projection to vocabulary logits.
\begin{verbatim}
return h @ self.wte.weight.T
\end{verbatim}

  \item[\texttt{@classmethod def from\_pretrained(cls, ...):}]  
    Load HuggingFace weights into this model.
\begin{verbatim}
hf    = HFModel.from_pretrained(model_name)
model = cls(our_cfg)
model.load_state_dict(filter_weights(hf.state_dict()))
return model
\end{verbatim}
\end{description}

\subsection{Optimizer Module (\texttt{optimizer.py})}

\begin{description}

  \item[\texttt{def \_\_init\_\_(self, params, lr, betas, eps, weight\_decay, correct\_bias):}]  
    Validates hyper‐parameters, stores defaults, and calls the base \texttt{Optimizer} constructor.


  \item[\texttt{def step(self, closure=None):}]  
    Performs one optimization step of AdamW:
    \begin{enumerate}
      \item[(1)] \textbf{(Optional)} call \texttt{closure()} to recompute loss.
      \item[(2)] Loop over each parameter group and each parameter \texttt{p}:
      \item[(3)] \textbf{State Initialization} (first time only):
\begin{verbatim}
if len(state)==0:
    state["step"]       = 0
    state["exp_avg"]    = torch.zeros_like(p.data)
    state["exp_avg_sq"] = torch.zeros_like(p.data)
\end{verbatim}
      \item[(4)] \textbf{Moment Updates}:
\begin{verbatim}
state["step"] += 1
exp_avg.mul_(beta1).add_(grad, alpha=1-beta1)
exp_avg_sq.mul_(beta2).addcmul_(grad, grad, value=1-beta2)
\end{verbatim}
      \item[(5)] \textbf{Bias‐Corrected Step Size}:
\begin{verbatim}
if correct_bias:
    bc1 = 1 - beta1**t
    bc2 = 1 - beta2**t
    step_size = lr * math.sqrt(bc2) / bc1
else:
    step_size = lr
\end{verbatim}
      \item[(6)] \textbf{Parameter Update}:
\begin{verbatim}
denom = exp_avg_sq.sqrt().add_(eps)
p.data.addcdiv_(exp_avg, denom, value=-step_size)
\end{verbatim}
      \item[(7)] \textbf{Decoupled Weight Decay}:
\begin{verbatim}
if weight_decay != 0:
    p.data.add_(p.data, alpha=-lr * weight_decay)
\end{verbatim}
    \end{enumerate}
    Returns the (possibly recomputed) loss.
\end{description}

  

\end{document}
