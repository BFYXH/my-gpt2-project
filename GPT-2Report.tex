\documentclass{article}
\usepackage[utf8]{inputenc}
\usepackage[T1]{fontenc}
\usepackage{amsmath, amsthm}
\usepackage{amsfonts}
\usepackage{algorithm}
\usepackage{algorithmic}
\usepackage{float} % 为 algorithm 的 [H] 选项提供支持
\usepackage[utf8]{inputenc}
\usepackage[T1]{fontenc}
\usepackage{xcolor}
\usepackage[colorlinks=true, urlcolor=red]{hyperref}
\usepackage{enumitem}
\usepackage{tikz}
\usepackage{booktabs}
\usepackage{multirow}
\usetikzlibrary{positioning}

\newtheorem{assumption}{Assumption}
\newtheorem{lemma}{Lemma}
% 定义 theorem 环境(按章节编号)
\newtheorem{theorem}{Theorem}[section]
% 以下环境共用 theorem 的编号
\theoremstyle{definition}
\newtheorem{definition}[theorem]{Definition}
\newtheorem{example}[theorem]{Example}
\newtheorem{remark}[theorem]{Remark}
\newtheorem{corollary}[theorem]{Corollary}
\newcommand{\Ito}{It\^o}
\newcommand{\sgn}{\operatorname{sgn}}
% 如果希望 Proposition 与 theorem 共享编号,则使用下面的语句
% \newtheorem{proposition}[theorem]{Proposition}
% 如果希望 Proposition 独立编号,则使用下面的语句
\newtheorem{proposition}{Proposition}
\title{Final Project Report: Building GPT-2}
\date{\today}

\begin{document}
    \maketitle
    \tableofcontents
  \section{Basic information}


  The proposal should have the following information at the top:
  \begin{itemize}
    \item \textbf{Title:} A Framework for Building and Fine-Tuning GPT-2 Models
    \item \textbf{Team members:} 
    \begin{itemize}
      \item Songlin Songlin (\texttt{zhousl24@tsinghua.edu.cn})
      
      \textbf{Contribution}: Doing theoratical research and writing project proposal, 
implementing the structure of GPT-2 within the process of
\href{https://github.com/AGIXLab/NLU-Course-Project-GPT-and-Downstream-Tasks/tree/main}{Stanford NLU Final Project Guidence},
which is the first part of our project.


      Sun Zhongwei(\texttt{sun-zw23@mails.tsinghua.edu.cn})

      Shibo Dai(\texttt{dsb24@mails.tsinghua.edu.cn})
    \end{itemize}
    \item \textbf{Custom or Default Project:} 
      Default Project---Building GPT-2
      % 若使用默认模板则填 “Default Project”
  \end{itemize}
  





  



  \section{Project description}

  \begin{itemize}
    \item \textbf{Main Goals:} 
      \begin{itemize}
        \item Use the codebase that is adapted from the final project of Stanford CS224n with pre-trained of GPT-2
        and implement some important components of the GPT-2 model to better understand its architecture.
        See \href{https://github.com/AGIXLab/NLU-Course-Project-GPT-and-Downstream-Tasks/tree/main}{this link}.
        \item Fine-tune the pretrained model on multiple downstream NLP tasks: 1)
        Sentiment Analysis, 2) Paraphrase Identification, and 3) Sonnet Generation.
        \item Train a mini-GPT model from scratch without loading any pretrained transformer weights (word embeddings are allowed).
      \end{itemize}
  
    \item \textbf{Downstream NLP Tasks:}
      \begin{itemize}
        \item Sentiment Analysis on IMDB reviews (binary classification).
        \item Paraphrase Detection on MRPC (sentence-pair classification).
        \item Sonnet Generation given a short Shakespearean prompt (autoregressive generation).
      \end{itemize}
  
    \item \textbf{Data Usage:}
      \begin{itemize}
        \item \emph{Pre-training:} None. The pre-trained data is provided within \emph{
          NLU-Course-Project-GPT-and-Downstream-Tasks}


        \item All we need to do is complete the implements of the missing block of algorithm such as Multi-headed Attention,
        Adam Optimization and position-wise Feed-Frward Network.
        \item \emph{Fine-tuning and Prediction for Sentiment Analysis:}
          \begin{itemize}
            \item Stanford Sentiment Treebank (SST) (8,544 train / 1,101 dev / 1,101 test).
            \item CFIMDB dataset (1,701 train / 245 dev / 488 test)
          \end{itemize}
        \item \emph{Fine-tuning and Prediction for Paraphrase Detection:}
          \begin{itemize}
            \item Quora dataset (141,506 train  / 20,215 dev / 40,431 test)
          \end{itemize}
        \item Test sets and methods are also provided in the e final project of
        Stanford CS224n.
      \end{itemize}
  
    \item \textbf{Methods:}
      \begin{itemize}
        \item Use the GPT-2 structure provided within \emph{
          NLU-Course-Project-GPT-and-Downstream-Tasks}: GPT-2 small (12 layers, 12 heads, 768-dim).
        \item Training with Adam, linear warmup (10 \% steps) + linear decay.
        \item Fine-tuning: add classification heads for IMDB/MRPC; use beam search for sonnet generation.
      \end{itemize}
  
    \item \textbf{Baselines:}
      \begin{itemize}
        \item For SA:
          \begin{itemize}
            \item Last Linear Layer for SST: Dev Accuracy: 0.462
            \item Full Model for SST: Dev Accuracy: 0.513
            \item Last Linear Layer for CFIMDB: Dev Accuracy: 0.861
            \item Full Model for CFIMDB: Dev Accuracy: 0.976
          \end{itemize}
        \item For PD:
        \begin{itemize}
          \item \textbf{IMDB Sentiment Analysis:}
              \item BERT‐base fine‐tuned (Devlin et al., 2019): 94.5\% accuracy.
              \item GPT-2 zero‐shot (Radford et al., 2019): 70.3\% accuracy.

            \item \textbf{MRPC Paraphrase Detection:}
              \item BERT‐base fine‐tuned (Devlin et al., 2019): 88.9\% accuracy, F1 = 89.2\%.
              \item GPT-2 zero‐shot (Radford et al., 2019): 65.1\% F1.
        \end{itemize}
        \item For Sonnet: Published BERT-base fine-tuned results (GLUE leaderboard).
      \end{itemize}
  
    \item \textbf{Evaluation Metrics:}
      \begin{itemize}
        \item See Baselines for SA.
        \item \emph{Accuracy} and \emph{F1} for IMDB and MRPC.
        \item \emph{CHRF score} for sonnet generation.
      \end{itemize}
  \end{itemize}

  \section{Songlin Zhou: Implementing GPT-2}
  \subsection{Folder and File Roles}

  \begin{description}
    \item[\texttt{\_\_pycache\_\_}]%
          A local snapshot of the \emph{official} HuggingFace
          GPT{\tiny -}2 weights; it is loaded on demand by
          \texttt{base\_gpt.py} to avoid an on-line download.
  
    \item[\texttt{Downstream-tasks/}]%
          Example fine–tuning code for text classification, summarisation,
          \emph{etc.}\;—\;not required in my assignment.
  
    \item[\texttt{models/}]  \hfill (central folder)
            \begin{description}
              \item[\texttt{\_\_pycache\_\_}]%
                     Compiled bytecode and cached tensors; reuse of the
                     pretrained weights.
  
              \item[\texttt{base\_gpt.py}]%
                     Generic “\emph{foundation}’’ class
                     (\texttt{GPTPreTrainedModel})\,: stores the
                     \texttt{config}, performs weight initialisation
                     and keeps track of global \texttt{dtype}. Every
                     concrete GPT variant inherits from this class.
  
              \item[\texttt{gpt2.py}]%
                     Instantiates the Transformer stack defined in
                     \texttt{base\_gpt.py}, wiring the attention blocks,
                     feed–forward layers, and the final LM head.
            \end{description}

    \item[\texttt{modules/}]
    

            \begin{description}
              \item[\texttt{attention.py}]%
                     Full implementation of \emph{Causal Self-Attention}
                     (multi–head, padding and causal masking, output
                     projection).
    
              \item[\texttt{gpt2\_layer.py}]%
                     Combines \texttt{attention.py} with
                     \emph{Pre-LayerNorm} and the feed-forward MLP to
                     form one Transformer block
                     (\texttt{Dropout $\rightarrow$ Dense $\rightarrow$
                     Residual} per layer).
          \end{description}
  
    \item[\texttt{test/}]%
          Minimal unit tests for GPT-2 model forward-pass correctness and for the
          \texttt{Adam} optimiser.
          \begin{verbatim}
            $ cd my-gpt2-project
            $ python3 test/optimizer_test.py
            Your GPT2 implementation is correct!
            $ python3 test/sanity_check.py
            Optimizer test passed!
            \end{verbatim}
            

          The following steps demonstrate the verification process for the GPT-2 implementation and optimizer:
  
    \item[\texttt{config.py}]%
          Default hyper-parameters (hidden size, \#layers, dropout
          probability, learning rate, \dots).
  
    \item[\texttt{optimizer.py}]%
          A clean implementation of the \texttt{Adam} algorithm.
    \item[\texttt{README.md}]%
          Step-by-step running instructions.
  \end{description}

  \begin{center}
    \fbox{\parbox{0.9\textwidth}{
        \centering
        \textbf{Code Repository} \\
        \url{https://github.com/Pinarinith/my-gpt2-project}
    }}
    \end{center}

  \subsection{Attention Module (\texttt{attention.py})}

\begin{description}
  \item[\texttt{def \_\_init\_\_(self, config):}] 
    Initializes the three linear projection layers for \textbf{query}, \textbf{key} and \textbf{value}, each of shape $[H]\!\to\![H]$, where $H=\text{hidden\_size}$.  Imports all relevant hyperparameters from \texttt{config} (e.g.\ \texttt{num\_attention\_heads}, \texttt{hidden\_size}, \texttt{attention\_dropout}).  Defines a \texttt{Dropout} layer that is applied to the \emph{normalized} attention scores following the original Transformer paper.  Although somewhat unusual, we empirically observe that this yields better generalisation.

  \item[\texttt{def transformer(self, x, linear\_layer):}]
    Given an input feature tensor 
    \[
      x\in\mathbb{R}^{B\times T\times H},
    \] 
    where $B$ is batch size, $T$ is sequence length and $H$ is hidden dimension, applies a single linear projection (\texttt{linear\_layer}) to obtain 
    \(\displaystyle \mathrm{proj}\in\mathbb{R}^{B\times T\times H}\), then reshapes into multiple heads:
    \[
      \text{proj}
        =\texttt{rearrange}(\text{proj},\,`\;b\;t\;(h\;d)\to b\;t\;h\;d',\;h=\text{num\_attention\_heads}),
    \]

    which yields a tensor of shape $[B,T,h,d]$ and enables parallel computation over $h$ attention heads of size $d=H/h$ for speed and efficiency.

  \item[\texttt{def attention(self, key, query, value, attention\_mask):}]
    Computes masked, scaled dot–product self–attention.  Given
    \[
      Q,K\in\mathbb{R}^{B\times h\times T\times d},\quad
      V\in\mathbb{R}^{B\times h\times T\times d},
    \]
    it first forms the unnormalized scores
    \[
      S = \frac{Q\,K^{\!\top}}{\sqrt{d_k}}\quad\Bigl[S\in\mathbb{R}^{B\times h\times T\times T}\Bigr],
    \]
    then applies the mask:
    \[
      S \leftarrow S + \text{attention\_mask},
    \]
    and finally normalises with a dropout to keep robustness:
    \[
      A = \mathrm{softmax}(S,\;\mathrm{dim}=-1),\quad
      A = \mathrm{Dropout}(A).
    \]
    The output context is
    \[
      \text{context} = A\,V\in\mathbb{R}^{B\times h\times T\times d},
    \]
    which is reshaped back to $[B,T,H]$ before returning.

    \item[\texttt{def forward(self, hidden\_states, attention\_mask):}]
    In the feed‐forward neural network we apply masking 
    of future information to prevent the model 
    from being influenced by future tokens during training, 
    which would degrade prediction performance. 
    Each \texttt{forward} call outputs an attention value 
    that becomes a subcomponent in \texttt{gpt2\_layer.py}.
    
\end{description}

\subsection{Layer Module (\texttt{gpt2\_layer.py})}

\begin{description}
  \item[\texttt{def \_\_init\_\_(self, config):}] 
    Initializes all sub–modules for one Transformer block:  
    \begin{itemize}
      \item Multi–head self–attention \(\bigl(\texttt{CausalSelfAttention}\bigr)\) parameters  
      \item Two linear layers for the position‐wise feed–forward network  
      \item Two \texttt{LayerNorm} instances for Pre–LN  
      \item Dropout probabilities from \texttt{config}  
    \end{itemize}

  \item[\texttt{def add(self, residual, sublayer\_out, dense\_layer, dropout):}]  
    \begin{itemize}
      \item Projects \texttt{sublayer\_out} back to the hidden dimension via \texttt{dense\_layer}.  
      \item Applies \texttt{dropout} for regularisation.  
      \item Adds the original \texttt{residual} tensor (residual connection).  
    \end{itemize}

  \item[\texttt{def forward(self, hidden\_states, attention\_mask):}]  
    Implements one full block in the Pre–LayerNorm style:
    \begin{enumerate}
      \item \textbf{Multi–Head Self–Attention}
        \begin{enumerate}
          \item[(1‐a)] \emph{Pre‐normalize}: apply \texttt{LayerNorm} to \texttt{hidden\_states}.
          \item[(1‐b)] \emph{Attention}: compute self–attention with \texttt{attention\_mask}.
          \item[(1‐c)] \emph{Dropout + Residual}: use \texttt{add(...)} to project, drop out, and add back the input.
        \end{enumerate}
      \item \textbf{Position‐wise Feed–Forward}
        \begin{enumerate}
          \item[(2‐a)] \emph{Pre‐normalize}: apply \texttt{LayerNorm} to the attention output.
          \item[(2‐b)] \emph{MLP}: apply \texttt{Linear} → \texttt{GELU} → \texttt{Linear}.
          \item[(2‐c)] \emph{Dropout + Residual}: use \texttt{add(...)} again to project, drop out, and add back.
        \end{enumerate}
    \end{enumerate}
\end{description}

  
\subsection{GPT2 Module Summary (\texttt{gpt2.py})}

\begin{description}

  \item[\texttt{def \_\_init\_\_(self, config):}]  
    Initializes embeddings, Transformer blocks, layer‐norm and weights.

  \item[\texttt{def embed(self, input\_ids):}]  
    Token + position lookup + dropout.

  \item[\texttt{def encode(self, hidden, mask):}]  
    Applies each layer with the extended attention mask.

  \item[\texttt{def forward(self, input\_ids, attention\_mask):}]  
    Embed → encode → final norm → select last token.
\begin{verbatim}
x    = self.embed(input_ids)
x    = self.encode(x, attention_mask)
x    = self.ln_f(x)
last = x[torch.arange(B), attention_mask.sum(1)-1]
return {'last_hidden_state': x, 'last_token': last}
\end{verbatim}

  \item[\texttt{def hidden\_state\_to\_token(self, h):}]  
    Weight–tie projection to vocabulary logits.
\begin{verbatim}
return h @ self.wte.weight.T
\end{verbatim}

  \item[\texttt{@classmethod def from\_pretrained(cls, ...):}]  
    Load HuggingFace weights into this model.
\begin{verbatim}
hf    = HFModel.from_pretrained(model_name)
model = cls(our_cfg)
model.load_state_dict(filter_weights(hf.state_dict()))
return model
\end{verbatim}
\end{description}

\subsection{Optimizer Module (\texttt{optimizer.py})}

\begin{description}

  \item[\texttt{def \_\_init\_\_(self, params, lr, betas, eps, weight\_decay, correct\_bias):}]  
    Validates hyper‐parameters, stores defaults, and calls the base \texttt{Optimizer} constructor.


  \item[\texttt{def step(self, closure=None):}]  
    Performs one optimization step of AdamW:
    \begin{enumerate}
      \item[(1)] \textbf{(Optional)} call \texttt{closure()} to recompute loss.
      \item[(2)] Loop over each parameter group and each parameter \texttt{p}:
      \item[(3)] \textbf{State Initialization} (first time only):
\begin{verbatim}
if len(state)==0:
    state["step"]       = 0
    state["exp_avg"]    = torch.zeros_like(p.data)
    state["exp_avg_sq"] = torch.zeros_like(p.data)
\end{verbatim}
      \item[(4)] \textbf{Moment Updates}:
\begin{verbatim}
state["step"] += 1
exp_avg.mul_(beta1).add_(grad, alpha=1-beta1)
exp_avg_sq.mul_(beta2).addcmul_(grad, grad, value=1-beta2)
\end{verbatim}
      \item[(5)] \textbf{Bias‐Corrected Step Size}:
\begin{verbatim}
if correct_bias:
    bc1 = 1 - beta1**t
    bc2 = 1 - beta2**t
    step_size = lr * math.sqrt(bc2) / bc1
else:
    step_size = lr
\end{verbatim}
      \item[(6)] \textbf{Parameter Update}:
\begin{verbatim}
denom = exp_avg_sq.sqrt().add_(eps)
p.data.addcdiv_(exp_avg, denom, value=-step_size)
\end{verbatim}
      \item[(7)] \textbf{Decoupled Weight Decay}:
\begin{verbatim}
if weight_decay != 0:
    p.data.add_(p.data, alpha=-lr * weight_decay)
\end{verbatim}
    \end{enumerate}
    Returns the (possibly recomputed) loss.
\end{description}

\section{Sun Zhongwei: Training and evaluating a mini-GPT model}

This part documents the process of training and evaluating a mini-GPT model for generating text in the style of William Shakespeare. The model was implemented using PyTorch and trained from scratch without pretrained transformer weights. We experimented with different architectures and hyperparameters to understand their impact on model performance.

\begin{center}
    \fbox{\parbox{0.9\textwidth}{
        \centering
        \textbf{Code Repository} \\
        \url{https://gitee.com/crazyysun/minigpt}
    }}
\end{center}

\subsection{Dataset Preparation}
The dataset used was \texttt{tinyshakespeare.txt} containing Shakespeare's works. The data was processed as follows:

\begin{itemize}
    \item Character-level tokenization was performed
    \item The dataset was split into:
    \begin{itemize}
        \item Training set: 80\% of the data
        \item Validation set: 10\% of the data
        \item Test set: 10\% of the data
    \end{itemize}
    \item Context length (block size) was set to 128 characters
\end{itemize}

The vocabulary size was determined to be \texttt{65} unique characters after processing.

\subsection{Model Architecture}
The mini-GPT model follows the standard Transformer architecture with the following components:

\begin{itemize}
    \item Token embeddings layer
    \item Positional embeddings
    \item Multiple transformer layers with self-attention
    \item Feed-forward networks
    \item Final linear projection to vocabulary size
\end{itemize}

\subsection{Training Procedure}
We experimented with multiple configurations as specified in the \texttt{config.json} file. Two representative configurations are highlighted below:

\subsubsection*{Configuration 1: Smaller Model}
\begin{itemize}
    \item Layers: 4
    \item Attention heads: 4
    \item Embedding dimension: 256
    \item Dropout: 0.1
    \item Learning rate: 3e-4
\end{itemize}

\subsubsection*{Configuration 2: Larger Model}
\begin{itemize}
    \item Layers: 8
    \item Attention heads: 8
    \item Embedding dimension: 512
    \item Dropout: 0.1
    \item Learning rate: 1e-4
\end{itemize}

Training was performed with:
\begin{itemize}
    \item Batch size: 64
    \item Epochs: 5
    \item Optimizer: AdamW
    \item Evaluation interval: Half of training set size
\end{itemize}

\subsection{Results}

The results of the two configurations are shown below, for more details see the training log on the repository website.

\subsubsection*{Configuration 1: Smaller Model}

The model architecture used the following parameters:
\begin{itemize}
    \item Transformer Layers: 4
    \item Attention Heads: 4
    \item Embedding Dimension: 256
    \item Dropout Rate: 0.1
    \item Learning Rate: 0.0003
\end{itemize}

\subsubsection*{Training Results}
The model achieved optimal performance after 20 epochs:
\begin{itemize}
    \item Best Validation Loss: 0.295
    \item Final Batch: 13,940
\end{itemize}

\subsubsection*{Test Performance}
The model was evaluated on a held-out test set with the following results:

\begin{table}[h]
\centering
\caption{Test Performance Metrics}
\label{tab:metrics}
\begin{tabular}{lc}
\toprule
\textbf{Metric} & \textbf{Value} \\
\midrule
Average Loss & 0.2950 \\
Token Accuracy & 91.45\% \\
Perplexity & 1.34 \\
Vocabulary Diversity & 0.3484 \\
Sequence Accuracy & 91.32\% \\
Exact Match Rate & 0.00\% \\
\bottomrule
\end{tabular}
\end{table}

\subsubsection*{Generated Text Samples}
The model was tested with different generation parameters:

\subsubsection*{Sample 1: Temperature=0.8, Top-k=50}
\begin{verbatim}
HAMLET: To be or not to be talk'd for, to accept it.

KING RICHARD II:
And buried, gentle Tyrrel?

TYRREL:
The grey-office with such a deceitful soul
That hath two them kings murder'd:
But they shall set up forward towards on their market-place;
Call me to the master?

Page:
No, no, no, no; for she hath praised his master.

TRANIO:
Sir, how now, sir! What's the matter? My dear inclinest maid
That wash'd my hands with thy sword sweet wives and with
Rainous with silver savage and talk of perpetual
To greet the traitor's na
\end{verbatim}

\subsubsection*{Sample 2: Temperature=0.5, Top-k=50}
\begin{verbatim}
HAMLET: To be or not to be endured!

GLOUCESTER:
I hope the king is not here pass'd for so fair?
She may year more, may betide to my suit;
And so, here comes Katharina, thy supper is:
Where is the best of Katharina must be.

SAMPSON:
The better for his father's death, and I will speak:
There is no more but strange.

Shepherd:
I cannot speak, Northumberland, I crave it,
When I shall not do it.

CLARENCE:
Then this is the duke my father, cry 'Come;'
O, 'alce on father, but not who came.

First Murderer:
Ay, stay we now no 
\end{verbatim}

\subsubsection*{Sample 3: Temperature=0.8, Top-k=0 (No restriction)}
\begin{verbatim}
HAMLET: To be or not to be so resolved
As vaults should be thine. To have heard your person;
'Tis more, thanks, my gracious lord.

DUKE VINCENTIO:

ISABELLA:

DUKE VINCENTIO:
Let's hear. O sir, ha! Away with him!

LUCIO:
A hundred moulderous leisure; the worst is full of
meat, and here be absent: it may be denied but my
believe in Vienna, is the grave of men alive?

PETRUCHIO:
It was an errand.

HORTENSIO:
Sir, fairly dream!

GREMIO:
Horse! a devil, a red son, or an apparent,
Of such post surfeits that we have always bef
\end{verbatim}

\subsubsection*{Analysis}
Key observations from the experiment:
\begin{itemize}
    \item The model achieved excellent token accuracy (91.45\%) while maintaining low perplexity (1.34)
    \item Higher temperature (0.8) produced more creative but less coherent text
    \item Lower temperature (0.5) resulted in more conservative but grammatically sound output
    \item Disabling Top-k sampling led to slightly less coherent generations
    \item The zero exact match rate confirms the model isn't simply memorizing text
\end{itemize}

Here is another example in a similar format, simulating results from a different model configuration:

\subsubsection*{Configuration 2: Medium Model}

The model architecture used the following parameters:
\begin{itemize}
    \item Transformer Layers: 8
    \item Attention Heads: 8
    \item Embedding Dimension: 512
    \item Dropout Rate: 0.1
    \item Learning Rate: 0.0001
\end{itemize}

\subsubsection*{Training Results}
The model achieved optimal performance after 25 epochs:
\begin{itemize}
    \item Best Validation Loss: 0.100
    \item Final Batch: 17,425
\end{itemize}

\subsubsection*{Test Performance}
The model was evaluated on a held-out test set with the following results:

\begin{table}[h]
\centering
\caption{Test Performance Metrics}
\label{tab:metrics_medium}
\begin{tabular}{lc}
\toprule
\textbf{Metric} & \textbf{Value} \\
\midrule
Average Loss & 0.1001 \\
Token Accuracy & 96.74\% \\
Perplexity & 1.11 \\
Vocabulary Diversity & 0.3766 \\
Sequence Accuracy & 96.46\% \\
Exact Match Rate & 0.00\% \\
\bottomrule
\end{tabular}
\end{table}

\subsubsection*{Generated Text Samples}
The model was tested with different generation parameters:

\subsubsection*{Sample 1: Temperature=0.8, Top-k=50}
\begin{verbatim}
HAMLET: To be or not to be spoke withal.
Here comes his servant: how now, Catesby,
What says he?

CATESBY:
My lord: he doth entreat your grace;
To visit him to-morrow or next day:
He is within, with two right reverend fathers,
Divinely bent to meditation;
And no worldly suit would he be moved,
To draw him from his holy exercise.

BUCKINGHAM:
Return, good Catesby, to thy lord again;
Tell him, myself, the mayor and citizens,
In deep designs and matters of great moment,
No less importing than our general good,
Are come to ha
\end{verbatim}

\subsubsection*{Sample 2: Temperature=0.5, Top-k=50}
\begin{verbatim}
HAMLET: To be or not to be spoke withal.
Here comes his servant: how now, Catesby,
What says he?

CATESBY:
My lord: he doth entreat your grace;
To visit him to-morrow or next day:
He is within, with two right reverend fathers,
Divinely bent to meditation;
And no worldly suit would he be moved,
To draw him from his holy exercise.

BUCKINGHAM:
Return, good Catesby, to thy lord again;
Tell him, myself, the mayor and citizens,
In deep designs and matters of great moment,
No less importing than our general good,
Are come to ha
\end{verbatim}

\subsubsection*{Sample 3: Temperature=0.8, Top-k=0 (No restriction)}
\begin{verbatim}
HAMLET: To be or not to be talked on, you shall bear more.

CORIOLANUS:
The gods begin to mock me. I, that now
Refused most princely gifts, am bound to beg
Of my lord general.

COMINIUS:
Take't; 'tis yours. What is't?

CORIOLANUS:
I sometime lay here in Corioli
At a poor man's house; he used me kindly:
He cried to me; I saw him prisoner;
But then Aufidius was within my view,
And wrath o'erwhelm'd my pity: I request you
To give my poor host freedom.

COMINIUS:
O, well begg'd!
Were he the butcher of my son, he should
Be fre
\end{verbatim}

\subsubsection*{Analysis}
Key observations from the experiment:
\begin{itemize}
    \item The model showed improved performance over the smaller variant with a token accuracy of 96.74\% and perplexity of 1.11
    \item At temperature 0.8, the model generated rich and contextually appropriate text while maintaining coherence
    \item Lower temperature settings produced more conservative but still stylistically consistent output
    \item Vocabulary diversity increased slightly compared to the smaller model, indicating better use of language variation
    \item Despite high sequence accuracy, the zero exact match rate suggests strong generalization rather than memorization
    \item Generated samples demonstrated understanding of Shakespearean style and structure
\end{itemize}


\subsection{Conclusion}
We successfully trained and evaluated mini-GPT models for Shakespearean text generation. The larger model with 8 layers and 512 embedding dimension performed better but at the cost of increased computational requirements. Future work could explore more sophisticated sampling techniques and larger context windows.



\end{document}
